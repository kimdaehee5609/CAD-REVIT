%	-------------------------------------------------------------------------------
% 
%
%	 Revit
%
%		2022년
%		8월
%		25일
%		목요일
%		첫작성
%
%
%
%
%
%
%
%	-------------------------------------------------------------------------------

	\documentclass[12pt, a3paper, landscape, oneside]{book}
%	\documentclass[12pt, a4paper, oneside]{book}
%	\documentclass[12pt, a4paper, landscape, oneside]{book}

		% --------------------------------- 페이지 스타일 지정
		\usepackage{geometry}
%		\geometry{landscape=true	}
		\geometry{top 		=10em}
		\geometry{bottom		=10em}
		\geometry{left		=8em}
		\geometry{right		=8em}
		\geometry{headheight	=4em} % 머리말 설치 높이
		\geometry{headsep		=2em} % 머리말의 본문과의 띠우기 크기
		\geometry{footskip		=4em} % 꼬리말의 본문과의 띠우기 크기
% 		\geometry{showframe}
	
%		paperwidth 	= left + width + right (1)
%		paperheight 	= top + height + bottom (2)
%		width 		= textwidth (+ marginparsep + marginparwidth) (3)
%		height 		= textheight (+ headheight + headsep + footskip) (4)



		%	===================================================================
		%	package
		%	===================================================================
%			\usepackage[hangul]{kotex}				% 한글 사용
			\usepackage{kotex}					% 한글 사용
			\usepackage[unicode]{hyperref}			% 한글 하이퍼링크 사용

		% ------------------------------ 수학 수식
			\usepackage{amssymb,amsfonts,amsmath}	% 수학 수식 사용
			\usepackage{mathtools}				% amsmath 확장판



		% ------------------------------ 
			\usepackage{fix-cm}	
			\usepackage[english]{babel}


		% ------------------------------ table 
			\usepackage{longtable}			%
			\usepackage{tabularx}			%
			\usepackage{tabu}				%





		%	--------------------------------------------------------------------------------------- 
		% 	toc 설정  - table of contents
		% 	
		% 	
		% 	----------------------------------------------------------------  문서 기본 사항 설정
			\setcounter{secnumdepth}{5} 		% 문단 번호 깊이
			\setcounter{tocdepth}{3} 			% 문단 번호 깊이 - 목차 출력시 출력 범위

			\setlength{\parindent}{0cm} 		% 문서 들여 쓰기를 하지 않는다.


		%	--------------------------------------------------------------------------------------- 
		% 	mini toc 설정
		% 	
		% 	
		% 	--------------------------------------------------------- 장의 목차  minitoc package
			\usepackage{minitoc}

			\setcounter{minitocdepth}{1}    	%  Show until subsubsections in minitoc
%			\setlength{\mtcindent}{12pt} 	% default 24pt
			\setlength{\mtcindent}{24pt} 	% default 24pt

		% 	--------------------------------------------------------- part toc
		%	\setcounter{parttocdepth}{2} 	%  default
			\setcounter{parttocdepth}{0}
		%	\setlength{\ptcindent}{0em}		%  default  목차 내용 들여 쓰기
			\setlength{\ptcindent}{0em}         


		% 	--------------------------------------------------------- section toc

			\renewcommand{\ptcfont}{\normalsize\rm} 		%  default
			\renewcommand{\ptcCfont}{\normalsize\bf} 	%  default
			\renewcommand{\ptcSfont}{\normalsize\rm} 	%  default


		%	=======================================================================================
		% 	tocloft package
		% 	
		% 	------------------------------------------ 목차의 목차 번호와 목차 사이의 간격 조정
			\usepackage{tocloft}

		% 	------------------------------------------ 목차의 내어쓰기 즉 왼쪽 마진 설정
			\setlength{\cftsecindent}{2em}			%  section

		% 	------------------------------------------ 목차의 목차 번호와 목차 사이의 간격 조정
			\setlength{\cftsecnumwidth}{2em}		%  section




		%	=======================================================================================
		% 	tikz package
		% 	
		% 	--------------------------------- 	
			\usepackage{tikz}%
			\usetikzlibrary{arrows,positioning,shapes}
			\usetikzlibrary{mindmap}			




% ------------------------------------------------------------------------------
% Begin document (Content goes below)
% ------------------------------------------------------------------------------
	\begin{document}
			\dominitoc
			\doparttoc			




			\title{신화}
			\maketitle


			\tableofcontents 		% 목차 출력
%			\listoffigures 			% 그림 목차 출력
			\cleardoublepage
			\listoftables 			% 표 목차 출력



% 	==============================================================================  chapter  Flow chart
	\clearpage	
	\section{바른 삼매}
	\pagestyle{empty}

		%	----------------------------------------------- mind map
		% 	하향 5개
		%	----------------------------------------------- mind map
		\begin{center}
		\tikz[	mindmap,
%				minimum size=3cm,
				text width=6em, 
				concept color=black!80,
				level 1/.style={level distance=5.5cm,sibling angle=90},  %찰나삼매
				level 2/.style={level distance=4.0cm,sibling angle=30},  %삼매 설명
				level 3/.style={level distance=4.0cm,sibling angle=40},  %색계선정
%				level 4/.style={level distance=5.0cm,sibling angle=30},  %색계초선
%				level 5/.style={level distance=5.0cm,sibling angle=30},
%				every annotation/.style={text width=20em, align=left}]
				concept/.append style={fill={none}} 
				]
				\node 	[concept] 				{바름 삼매}
						[clockwise from=00, every concept/.style={minimum size=3cm}]
				child	{					node[concept] 	{찰나 삼매}
						[clockwise from=-90]
						child	{		node [annotation] {특정한 대상에만 집중하는 것이 아니라 현재 일어나는  대상에 순간 순간 집중하는 삼매를 말한다 }
							}
					}
				child	{	node[concept] 	{근접 삼매}
						[clockwise from=-90]
						child	{		node [annotation] {몰입 삼매에 들러가기 직전이나 직후의 마음 상태를 말한다}
							}
					}
				child	{				node[concept] 	{몰입 삼매}
									[clockwise from=180]
						child	{		node [annotation] {오직 하나의 대상만을 기억하여 알아차림으로써 마음이 하나의 대상에 완전히 몰입된 삼매를 말한다 }
							}
%									[clockwise from=45]
							child	{	node[concept] 	{색계선정}
%									[clockwise from=90]
%							child		[grow=45, level distance=8.0cm ]{ node[concept] {초기 균열	}}  
%							child		[grow=25, level distance=8.0cm ]{ node[concept] {침하수축균열	}}    
%							child		[grow=00, level distance=8.0cm ]{ node[concept] {소성수축균열	}}   
%							child		[grow=-25, level distance=8.0cm ]{ node[concept] {침하균열	}} 
%							child		[level distance=8.0cm ]{ node[concept] {초기 균열	}}  
%							child		[level distance=8.0cm ]{ node[concept] {침하수축균열	}}    
%							child		[level distance=8.0cm ]{ node[concept] {소성수축균열	}}   
%							child		[level distance=8.0cm ]{ node[concept] {침하균열	}} 
							child		{ node[concept] {색계 초선}}  
							child		{ node[concept] {색계 이선}}    
							child		{ node[concept] {색계 삼선}}   
							child		{ node[concept] {색계 사선}} 
%							child		{ node {색계 초선}}  
%							child		{ node {색계 이선}}    
%							child		{ node {색계 삼선}}   
%							child		{ node {색계 사선}} 
								}
%									[clockwise from=-45]
							child	{	node[concept] 	{무색계선정}
%									[clockwise from=45]
%							child		[grow=45, level distance=8.0cm ]{ node[concept] {초기 균열	}}  
%							child		[grow=25, level distance=8.0cm ]{ node[concept] {침하수축균열	}}    
%							child		[grow=00, level distance=8.0cm ]{ node[concept] {소성수축균열	}}   
%							child		[grow=-25, level distance=8.0cm ]{ node[concept] {침하균열	}} 
%							child		[level distance=8.0cm ]{ node[concept] {초기 균열	}}  
%							child		[level distance=8.0cm ]{ node[concept] {침하수축균열	}}    
%							child		[level distance=8.0cm ]{ node[concept] {소성수축균열	}}   
%							child		[level distance=8.0cm ]{ node[concept] {침하균열	}} 
							child		{ node[concept] {공무변처}}  
							child		{ node[concept] {식무변처}}    
							child		{ node[concept] {무소변처}}   
							child		{ node[concept] {비상비비상처}} 
								}
					};
		\end{center}

% 	==============================================================================  chapter  Flow chart
%	\clearpage	
	\section{바른 삼매}
	\pagestyle{empty}

		%	----------------------------------------------- mind map
		% 	하향 5개
		%	----------------------------------------------- mind map
		\begin{center}
		\tikz[	mindmap,
				grow cyclic,
				text width=6em, 
				concept color=black!10,
				level 1/.style={level distance=3.0cm,minimum size=3cm,},
				level 2/.style={level distance=3.0cm,minimum size=2cm,},
				level 3/.style={level distance=3.0cm,minimum size=2cm,},
				level 4/.style={level distance=3.0cm,minimum size=2cm,},
				concept/.append style={fill={none}} 
				]
				\node 	[concept,minimum size=4cm]	{바른 삼매}
%						[every concept/.style={minimum size=2cm}] 	% 여기서 하부 콥셉의 크기를 지정
				child	[grow=90, level distance=5.0cm ]	
				{					node[concept] 	{NAS}
					child				[grow=45, level distance=5.0cm ]
					{				node[concept] 	{https://\\quickconnect.to\\/lf0000}
						child			[grow=0, level distance=4.0cm ]	
						{			node[concept] 	{h 010\\3839\\5609}
							child		[grow=0, level distance=4.0cm ]	
							{		node[concept] 	{!김대희\\38395609}
							} 
						} 
					}
					child	[grow=0, level distance=4.0cm ]	
					{				node[concept] 	{guest}
						child			[grow=0, level distance=4.0cm ]	
						{			node[concept] 	{lf0000}
							child		[grow=0, level distance=4.0cm ]	
							{		node[concept] 	{00 00 00}
							} 
						} 
					} 
				}
				child	[grow=-10, level distance=7.0cm ]	
				{					node[concept] 	{네트워크}
					child 				[grow=30, level distance=6.0cm ]
					{				node[concept] 	{색계 선정}
						child			[grow=00, level distance=4.0cm ]
						{			node[concept] 	{색계 초선}} 
						child			[grow=00, level distance=4.0cm ]
						{			node[concept] 	{색계 초선}} 
					}
					child 				[grow=00, level distance=4.0cm ]
					{				node[concept] 	{무색계 선정}
						child 			[grow=45, level distance=8.0cm ]{ node[concept] {초기 균열	}}  
						child 			[grow=25, level distance=8.0cm ]{ node[concept] {침하수축균열	}}    
						child 			[grow=00, level distance=8.0cm ]{ node[concept] {소성수축균열	}}   
						child 			[grow=-25, level distance=8.0cm ]{ node[concept] {침하균열	}} 
					}
%					child				[grow=-30, level distance=6.0cm ]
%					{				node[concept] 	{LFNAS}
%						child			[grow=00, level distance=4.0cm ]
%						{			node[concept] 	{admin}
%							child		[grow=00, level distance=4.0cm ]
%							{		node[concept] 	{lf 0000}} 
%						} 
%					}
				}
				child	[grow=-90, level distance=7.0cm ]	
				{					node[concept] 	{web hard}
				};
				
		\end{center}



% 	==============================================================================  chapter  Flow chart
	\section{작업중}
	\pagestyle{empty}
	
		\begin{center}
		\begin{tikzpicture}[	mindmap,
					    	every node/.style=concept,
					    	concept color=black!10,
						concept/.append style={fill={none}},
						grow cyclic,
%					    level 1/.append style={level distance=4.0cm,sibling angle=360},
					    level 1/.append style={level distance=6.0cm},
					    level 2/.append style={level distance=6.5cm}
					    level 3/.append style={level distance=6.5cm}
					    ]
		  \node 	[root concept] 					{	삼매 } % root
				[clockwise from=0]
				%
			    	child 	[level distance=12cm , grow=0]	{ node {찰나 삼매}
					[clockwise from=0]
					child		[level distance=6cm]	{ node {공기량} 
						child 					{ node {여기가 어디 2} }  
						child 					{ node {여기가 어디 2} }   
						child 					{ node {여기가 어디 2} }  }
					child 		[level distance=6cm]	{ node {재료분리} 
						child 					{ node {여기가 어디 2} }  
						child 					{ node {여기가 어디 2} }   
						child 					{ node {여기가 어디 2} }  } 
					child 		[level distance=6cm]	{ node {응결} 
						child 	[grow=00]			{ node {1} }  
						child 	[grow=30]			{ node {2} }   
						child 	[grow=60]			{ node {3} }  } 
					child 		[level distance=6cm]	{ node {균열		} 
						child 					{ node {초기 균열} 	}  
						child 					{ node {침하수축균열	} }   
						child 					{ node {소성수축균열	} }   
						child 					{ node {침하균열	} } } }
				%
			    	child [grow=-90 ]			 	{ node {근접 삼매}
					[clockwise from=0]
					child 						{ node {여기가 어디 1} } 
					child 						{ node {여기가 어디 2} }  }
				%				
				child [grow=180]				 	{ node [concept]{몰입 삼매}
					child 	[grow=90]					{ node [concept]{색계 선정} 
						child 					{ node [concept]{색계 초선} } 	
						child 					{ node [concept]{색계 이선} } 	
						child 					{ node [concept]{색계 삼선} } 	
						child 					{ node [concept]{색계 사선} } }
					child 	[grow=-90]			{ node [concept]{무색계 선정} 
						child 					{ node [concept]{색계 초선} } 	
						child 					{ node [concept]{색계 이선} } 	
						child 					{ node [concept]{색계 삼선} } 	
						child 					{ node [concept]{색계 사선} } } };
		\end{tikzpicture}
		\end{center}



	\clearpage

		\tikz[mindmap,concept color=red!50]
					\node [concept] 	{Root concept}
					child[grow=right] 	{node[concept] {Child concept}};


		\paragraph{every mindmap style} \hfill \\
		\tikz[every mindmap,concept color=red!50]
					\node [concept] 	{Root concept}
					child[grow=right] 	{node[concept] {Child concept}};

		\paragraph{mindmap style} \hfill \\
		\tikz[mindmap,concept color=red!50]
					\node [concept] 	{Root concept}
					child[grow=right] 	{node[concept] {Child concept}};

		\paragraph{large mindmap style} \hfill \\
		\tikz[large mindmap,concept color=red!50]
					\node [concept] 	{Root concept}
					child[grow=right] 	{node[concept] {Child concept}};

		\paragraph{huge mindmap style} \hfill \\
		\tikz[huge mindmap,concept color=red!50]
					\node [concept] 	{Root concept}
					child[grow=right] 	{node[concept] {Child concept}};


	\clearpage


		\paragraph{concept} \hfill \\
		\tikz[mindmap,concept color=red!10] \node [concept] {강도} ;
		\tikz[mindmap,concept color=red!10] \node [concept] {내구성} ;
		\tikz[mindmap,concept color=black!10] \node [concept] {수밀성} ;
		\tikz[mindmap,concept color=black!10] \node [concept] {균열\\저항성} ;
		\tikz[mindmap,concept color=black!10] \node [concept] {강재\\보호} ;


		\paragraph{extra concept} \hfill \\
		\tikz[mindmap,concept color=red!10] \node [extra concept] {강도} ;
		\tikz[mindmap,concept color=red!10] \node [extra concept] {강도} ;
		\tikz[mindmap,concept color=red!10] \node [extra concept] {내구성} ;
		\tikz[mindmap,concept color=red!10] \node [extra concept] {수밀성} ;
		\tikz[mindmap,concept color=red!10] \node [extra concept] {균열\\저항성} ;
		\tikz[mindmap,concept color=red!10] \node [extra concept] {강재\\보호} ;

		\tikz[mindmap,concept color=black!10] \node [extra concept, color=black!10] {강도} ;



		\paragraph{color} \hfill  \\

		\tikz[mindmap, concept color=black!10, text=red]	%
				\node [concept] {\Large 강도} 
				child[concept color=red!5, grow=00] {node[concept] {\Large 강도-1} } ;


		\tikz[mindmap, concept color=black!80, text=white]	%
				\node [concept] {\Large 강도} 
				child[concept color=black!80, text=white,  grow=00] {node[concept] {강도-1} } ;





		\paragraph{concept size} \hfill \\
		\tikz[mindmap,minimum size=0.5cm, concept color=red!10] 		\node [concept] {강도} ;
		\tikz[mindmap,minimum size=1cm, concept color=red!10] 		\node [concept] {1cm} ;
		\tikz[mindmap,minimum size=2cm, concept color=red!10] 		\node [concept] {2cm} ;
		\tikz[mindmap,minimum size=3cm, concept color=black!10]		\node [concept] {3cm 수밀성} ;
		\tikz[mindmap,minimum size=4cm, concept color=black!10] 		\node [concept] {4cm 균열\\저항성} ;
		\tikz[mindmap,minimum size=5cm, concept color=black!10] 		\node [concept] {5cm 강재\\보호} ;
		\tikz[mindmap,minimum size=4cm, concept color=black!10] 		\node [concept] {\Large 균열\\저항성} ;

		\paragraph{every mindmap size} \hfill \\
		\tikz[every mindmap,minimum size=0.5cm, concept color=red!10] 	\node [concept] {0.5cm} ;
		\tikz[every mindmap,minimum size=1cm, concept color=red!10] 		\node [concept] {1cm} ;
		\tikz[every mindmap,minimum size=2cm, concept color=red!10] 		\node [concept] {2cm} ;
		\tikz[every mindmap,minimum size=3cm, concept color=black!10]		\node [concept] {수밀성} ;
		\tikz[every mindmap,minimum size=4cm, concept color=black!10]		\node [concept] {\Large 균열\\저항성} ;
		\tikz[every mindmap,minimum size=5cm, concept color=black!10]		\node [concept] {강재\\보호} ;




		\paragraph{grow} \hfill  \\

		\tikz[mindmap, concept color=black!10]	%
				\node [concept] {강도}  child[concept color=red!5, grow=00] {node[concept] {강도-1} } ;

		\tikz[mindmap, concept color=black!10]	%
				\node [concept] {강도} 	child[concept color=red!5, grow=000] {node[concept] {강도-1} }
									child[concept color=red!5, grow=030] {node[concept] {강도-2} }
									child[concept color=red!5, grow=060] {node[concept] {강도-3} }
									child[concept color=red!5, grow=090] {node[concept] {강도-4} }
									child[concept color=red!5, grow=120] {node[concept] {강도-5} };









		\begin{tikzpicture}[	mindmap,
					    every node/.style=concept,
					    concept color=black!20,
					    grow cyclic,
					    level 1/.append style={level distance=4.0cm,sibling angle=90},
					    level 2/.append style={level distance=4.5cm,sibling angle=45}
					    ]
		  \node [root concept] 
			{	최고의 가치 } % root
				%
			    	child [concept color=black!10] 	%
										{ node {일}
				child 						{ node {여기가 어디 1} } 
				child 						{ node {여기가 어디 2} }  }
				%
				child [concept color=blue!10] 	{ node {직장}
				child 						{ node {Turing Machines} }
				child 						{ node {Random-Access Machines} }
				child 						{ node {Random-Access Machines} }
				child 						{ node {Random-Access Machines} }  }
				%				
				child [concept color=blue!10] 	{ node {가정}
				child 						{ node {Turing Machines} }
				child 						{ node {Random-Access Machines} }
				child 						{ node {Random-Access Machines} }
				child 						{ node {Random-Access Machines} }  };
		\end{tikzpicture}




	\clearpage



% ------------------------------------------------------------------------------
% End document
% ------------------------------------------------------------------------------
\end{document}



% =================================================================================================== Part 혼화 재료

